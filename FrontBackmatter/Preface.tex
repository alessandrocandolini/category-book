\chapter*{Reflections on category theory}
\markboth{\spacedlowsmallcaps{Preface}}{\spacedlowsmallcaps{Preface}}
\addcontentsline{toc}{chapter}{\tocEntry{Preface}}


These notes provide an exploration of concepts and ideas in category theory, with special emphasis on various flavours of monoidal categories, with applications to algebra, non-relativistic quantum mechanics, and applied functional programming. 

Category theory has recently gained prominence in several areas of pure and applied mathematics, physics, and computer science. 
Due to its extreme generality and versatility, category theory provides a powerful unifying language through which a wide variety of topics can be recast and understood. 
However, a question arises: is this merely an exercise in purely theoretical interest, or does it offer compelling advantages over more traditional approaches? 
To explore and stimulate discussion around this question, these notes provide a gallery of examples illustrating how the lens of category theory can unlock new ways of reasoning, provide a bird's-eye view that connects seemingly unrelated areas, and inspire fresh and insightful perspectives on various domains. 
While at first glance the discipline might seem impenetrable, the economy of thought it fosters can, in the long term, prove to be a rewarding investment.
Indeed, category theory serves as a conceptual and theoretical framework for expressing, identifying, and reasoning about commonalities across disparate domains, stripping away accidental complexity, noise, ``implementation details'', and the unnecessary burden of traditional formalisms, allowing core aspects and essential properties to emerge.

Consider, for instance, the operatorial formulation of non-relativistic quantum mechanics, pioneered by von Neumann and based on the spectral theory of linear self-adjoint operators on Hilbert spaces. (An alternative but equivalent formulation, the path integral approach developed by R.~Feynman, provides valuable physical insights but arguably plays a more prominent role in relativistic quantum field theories.) Despite the undeniable usefulness and effectiveness of the operatorial formalism in describing and calculating properties of various quantum systems, the language of Hilbert spaces can be overly verbose and clumsy for modeling, navigating, and characterising the general structure and universality of certain results.
For example, the no-cloning theorem, which is commonly derived using ad hoc technical constructs such as the Cauchy--Schwarz inequality, arises trivially from the lack of Cartesian structure in dagger compact categories. 
In categorical terms, the absence of diagonal morphisms inherently forbids the duplication of arbitrary states.
Cartesian categories are insufficient to fully describe quantum systems and, similarly, fail to capture effectful computations (\eg, applicative or monadic programs) in strongly typed, purely functional programming languages such as Haskell.

This is not to say that category theory is always suitable for every kind of task. 
It is important to resist the temptation to overuse category theory when more practical, ``down-to-earth'' encodings might provide superior solutions, or category theory might provide no solutions at all.
Sometimes, when applying quantum mechanics or computer science concepts to specific systems or programs, the goal is precisely to address the particularities of those systems (\eg, solving a specific differential equation, computing a probability distribution, or implementing an algorithm efficiently).
In such cases, category theory might not be the best framework for delving into the nitty-gritty details, as these are often intentionally abstracted away and rendered invisible within the categorical construction. 
However, adopting the perspective of a category theorist provides a complementary viewpoint from which certain properties can be more naturally expressed, explored, and discovered.


\smallskip

\noindent\textsw{\myLocation, \myTime}


\begin{flushright}
        \myName
\end{flushright}

